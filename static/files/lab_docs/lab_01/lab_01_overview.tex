% Options for packages loaded elsewhere
\PassOptionsToPackage{unicode}{hyperref}
\PassOptionsToPackage{hyphens}{url}
%
\documentclass[
]{article}
\usepackage{lmodern}
\usepackage{amssymb,amsmath}
\usepackage{ifxetex,ifluatex}
\ifnum 0\ifxetex 1\fi\ifluatex 1\fi=0 % if pdftex
  \usepackage[T1]{fontenc}
  \usepackage[utf8]{inputenc}
  \usepackage{textcomp} % provide euro and other symbols
\else % if luatex or xetex
  \usepackage{unicode-math}
  \defaultfontfeatures{Scale=MatchLowercase}
  \defaultfontfeatures[\rmfamily]{Ligatures=TeX,Scale=1}
\fi
% Use upquote if available, for straight quotes in verbatim environments
\IfFileExists{upquote.sty}{\usepackage{upquote}}{}
\IfFileExists{microtype.sty}{% use microtype if available
  \usepackage[]{microtype}
  \UseMicrotypeSet[protrusion]{basicmath} % disable protrusion for tt fonts
}{}
\makeatletter
\@ifundefined{KOMAClassName}{% if non-KOMA class
  \IfFileExists{parskip.sty}{%
    \usepackage{parskip}
  }{% else
    \setlength{\parindent}{0pt}
    \setlength{\parskip}{6pt plus 2pt minus 1pt}}
}{% if KOMA class
  \KOMAoptions{parskip=half}}
\makeatother
\usepackage{xcolor}
\IfFileExists{xurl.sty}{\usepackage{xurl}}{} % add URL line breaks if available
\IfFileExists{bookmark.sty}{\usepackage{bookmark}}{\usepackage{hyperref}}
\hypersetup{
  pdftitle={Introduction to EES 3310/5310 Labs},
  hidelinks,
  pdfcreator={LaTeX via pandoc}}
\urlstyle{same} % disable monospaced font for URLs
\usepackage[margin=1in]{geometry}
\usepackage{graphicx,grffile}
\makeatletter
\def\maxwidth{\ifdim\Gin@nat@width>\linewidth\linewidth\else\Gin@nat@width\fi}
\def\maxheight{\ifdim\Gin@nat@height>\textheight\textheight\else\Gin@nat@height\fi}
\makeatother
% Scale images if necessary, so that they will not overflow the page
% margins by default, and it is still possible to overwrite the defaults
% using explicit options in \includegraphics[width, height, ...]{}
\setkeys{Gin}{width=\maxwidth,height=\maxheight,keepaspectratio}
% Set default figure placement to htbp
\makeatletter
\def\fps@figure{htbp}
\makeatother
\setlength{\emergencystretch}{3em} % prevent overfull lines
\providecommand{\tightlist}{%
  \setlength{\itemsep}{0pt}\setlength{\parskip}{0pt}}
\setcounter{secnumdepth}{-\maxdimen} % remove section numbering
\usepackage{mathptmx}
\usepackage{float}
\usepackage{booktabs}
\usepackage[version=4]{mhchem}

\newcommand{\COO}{\ce{CO2}}
\newcommand{\methane}{\ce{CH4}}
\newcommand{\degC}{^\circ \mathrm{C}}
\newcommand{\degF}{^\circ \mathrm{F}}
\newcommand{\water}{\mathrm{H_2O}}
\newcommand{\carb}{\ce{CO3^2-}}
\newcommand{\bicarb}{\ce{HCO3-}}
\newcommand{\carbonic}{\ce{H2CO3}}
\newcommand{\Hplus}{\ce{H+}}
\newcommand{\OH}{\ce{OH-}}
\newcommand{\silica}{\ce{SiO2}}
\newcommand{\calcite}{\ce{CaCO3}}
\newcommand{\Caplus}{\ce{Ca^2+}}
\newcommand{\silicate}{\ce{SiO3^2-}}
\newcommand{\CaSi}{\ce{CaSiO3}}
\newcommand{\pH}{p\ce{H}}
\newcommand{\permil}{\permille}

\title{Introduction to EES 3310/5310 Labs}
\author{}
\date{\vspace{-2.5em}2020-01-06 06:00:00}

\begin{document}
\maketitle

{
\setcounter{tocdepth}{2}
\tableofcontents
}
\hypertarget{overview-of-ees-33105310-labs}{%
\section{Overview of EES 3310/5310
Labs}\label{overview-of-ees-33105310-labs}}

The laboratories in this course are computational. My goals for the
laboratory section are:

\begin{enumerate}
\def\labelenumi{\arabic{enumi}.}
\tightlist
\item
  Learn about best practices for \emph{reproducible research} and get
  experience applying tools and methods for making sure that your
  research is reliable, reproducible, and trustworthy. We will focus on
  research about climate science and climate and energy policy, but the
  methods and tools we will use are widely used in all kinds of research
  in natural and social sciences and also in the private sector.
\item
  Get experience working with real data: download and analyze data and
  report the results of your analysis.
\item
  Get experience working with computer models of different aspects of
  the climate system. Learn how to use models to do science, how to
  analyze and interpret the results of model simulations, and how to
  write reports about research using computational models.
\end{enumerate}

\hypertarget{general-policies}{%
\subsection{General Policies}\label{general-policies}}

\begin{itemize}
\tightlist
\item
  There will be \textbf{two} web resources for each lab:

  \begin{itemize}
  \tightlist
  \item
    \textbf{Documentation} that describes the lab and tells you what you
    need to do to prepare and what you will do in the lab class.
  \item
    \textbf{An assignment} that provides a template you will use in
    carrying out the lab and writing it up. The assignment will consist
    of a web link for you to click on to accept the assignment in GitHub
    Classroom. After you accept the assignment, GitHub Classroom will
    copy the assignment into your own GitHub Classroom account. You will
    then clone the assignment from GitHub to your own computer or a
    computer in the lab classroom to work on it.
  \end{itemize}

  Both the documentation and the assignment will be posted to the course
  web site at \url{https://ees3310.jgilligan.org/schedule/} at least one
  week before the lab class.
\item
  \textbf{Before} coming to lab:

  \begin{itemize}
  \tightlist
  \item
    Be sure to read the documentation for that week's lab.

    \begin{itemize}
    \tightlist
    \item
      Because the first lab of the semester is on the first day of
      class, this is an exception and you should read the documentation
      for the first lab during the following week, but \textbf{do be
      sure to read the documents for lab \#2 before you come to lab on
      Monday, Jan.~13}
    \end{itemize}
  \item
    Accept the assignment on GitHub Classroom (click on the Assignment
    link in the course web site). If you will be bringing your own
    computer to the lab, you may want to clone the assignment onto your
    own computer before you come to lab, but that's not strictly
    necessary. If you will be using one of the computers in the lab
    classroom, you can clone the assignment when you log in at the
    beginning of class.
  \end{itemize}
\item
  In the first lab, on Monday, Jan.~6, Ms.~Best will explain all of the
  different software we will be using and will walk you through all the
  steps of using Git to work with your assignments.
\end{itemize}

\hypertarget{schedule-for-the-semester}{%
\subsection{Schedule for the Semester}\label{schedule-for-the-semester}}

The semester is divided roughly in half.

\hypertarget{first-half-of-the-semester}{%
\subsubsection{First Half of the
Semester:}\label{first-half-of-the-semester}}

In the first half, the readings, class sessions, and laboratories will
focus on understanding the science of how the earth's climate system
works and how human perturbations to the environment may affect the
climate.

The weekly labs during the first half of the semester will initiially
focus on exercises from the book \emph{Global Warming: Understanding the
Forecast} and then on Feb.~17--Mar.~16 you will work in pairs to develop
your own project to investigate a question about the climate using the
computer models and/or data from the major climatic data archives.

\hypertarget{extended-lab-project}{%
\paragraph{Extended Lab Project}\label{extended-lab-project}}

The project has the following important dates:

\begin{itemize}
\tightlist
\item
  You and your partner will get your research topic approved by
  Prof.~Gilligan or Ms.~Best by \textbf{Friday, February 21}.
\item
  You and your partner will turn in a written report on
  \textbf{Thursday, March 12}.
\item
  You and your partner will give an oral presentation of your research
  project in the lab on \textbf{Monday, March 16}.
\end{itemize}

\hypertarget{second-half-of-the-semester}{%
\subsubsection{Second Half of the
Semester}\label{second-half-of-the-semester}}

In the second half of the semester, you will analyze data on the
economies and energy use of different countries around the world and use
these to analyze different policy options for reducing greenhouse gas
emissions. The labs will begin with exercises that follow the analyses
you will be reading about in the book, \emph{The Climate Fix}, and then
you and a partner will conduct a detailed analysis of policy options for
a country of your choice to make a transition to a cleaner energy
supply.

\hypertarget{policy-analysis-for-a-country}{%
\paragraph{Policy Analysis for a
Country}\label{policy-analysis-for-a-country}}

The policy analysis project has the following important dates:

\begin{itemize}
\tightlist
\item
  You and your partner will turn in a written policy analysis report on
  \textbf{Friday, April 17}.
\item
  You and your partner will present the results of your policy analysis
  in the lab on \textbf{Monday, April 20}.
\end{itemize}

\hypertarget{laboratory-classroom}{%
\section{Laboratory Classroom}\label{laboratory-classroom}}

The laboratory will meet in
\href{https://as.vanderbilt.edu/vuit/computer_services/facilities/Wilson.php}{the
Computer Classroom in 120 Wilson Hall}. This classroom is equipped with
computers that have all the software you need for this class. You are
also free to bring your own laptop to the lab. Detailed instructions for
installing R, RStudio, git, and (optionally) the LaTeX typesetting
system on your computer (for Windows, MacOs, and Linux) are available on
the course web site at \url{https://ees3310.jgilligan.org/tools/}.

The computer laboratory is accessible using your Vanderbilt ID card 6:00
AM--1:00 AM 7 days a week except when other classes are meeting there.
You can check the availability of the computer lab using Vanderbilt's
\href{https://emscampus.app.vanderbilt.edu/VirtualEms/LocationDetails.aspx?data=XHoo9AMVGMQbSraiXQjvGUDuKA574KqHNLkvgXI29xak6HZ4Mu3JOk0wo\%2fFGuBk+\#room-availability}{VirtualEMS
web app}.

\hypertarget{software-tools}{%
\section{Software Tools}\label{software-tools}}

We will use four principal software tools for this class. All four are
free and available for Windows, MacOS, and Linux. Detailed instructions
for downloading and installing them are available on the course web site
at \url{https://ees3310.jgilligan.org/tools/}:

\begin{enumerate}
\def\labelenumi{\arabic{enumi}.}
\item
  R: statistical analysis software
\item
  RStudio: a user-friendly interface to R, which makes it much easier to
  use.
\item
  Git: a tool for keeping track of revisions in computer code and
  documents and coordinating working together with other people on a
  project.
\item
  LaTeX: This is strictly optional. It is a typesetting package that
  allows you to produce PDF (acrobat) documents from RStudio. If you
  choose not to install LaTeX, you can still produce HTML (web) and DOCX
  (Microsoft Word) documents. The downside is that a full LaTeX
  distribution can be quite large (several hundred megabytes), so if you
  are running low on disk space, you might not want to install it.

  LaTeX is installed on the lab computers, so you will always have the
  option of producing PDF documents there, even if you don't install it
  on your own computer.
\end{enumerate}

All of this software is installed on the lab computers in Wilson 120, so
you do not need to install it on your own computer, but you will need it
to do the data analysis and write up your lab reports, so it will
probably be convenient for you to install your own copies.

\end{document}
