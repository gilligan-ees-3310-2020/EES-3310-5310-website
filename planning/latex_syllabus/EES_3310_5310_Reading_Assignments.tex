% Options for packages loaded elsewhere
\PassOptionsToPackage{unicode}{hyperref}
\PassOptionsToPackage{hyphens}{url}
%
\documentclass[
]{article}
\usepackage{lmodern}
\usepackage{amssymb,amsmath}
\usepackage{ifxetex,ifluatex}
\ifnum 0\ifxetex 1\fi\ifluatex 1\fi=0 % if pdftex
  \usepackage[T1]{fontenc}
  \usepackage[utf8]{inputenc}
  \usepackage{textcomp} % provide euro and other symbols
\else % if luatex or xetex
  \usepackage{unicode-math}
  \defaultfontfeatures{Scale=MatchLowercase}
  \defaultfontfeatures[\rmfamily]{Ligatures=TeX,Scale=1}
\fi
% Use upquote if available, for straight quotes in verbatim environments
\IfFileExists{upquote.sty}{\usepackage{upquote}}{}
\IfFileExists{microtype.sty}{% use microtype if available
  \usepackage[]{microtype}
  \UseMicrotypeSet[protrusion]{basicmath} % disable protrusion for tt fonts
}{}
\makeatletter
\@ifundefined{KOMAClassName}{% if non-KOMA class
  \IfFileExists{parskip.sty}{%
    \usepackage{parskip}
  }{% else
    \setlength{\parindent}{0pt}
    \setlength{\parskip}{6pt plus 2pt minus 1pt}}
}{% if KOMA class
  \KOMAoptions{parskip=half}}
\makeatother
\usepackage{xcolor}
\IfFileExists{xurl.sty}{\usepackage{xurl}}{} % add URL line breaks if available
\IfFileExists{bookmark.sty}{\usepackage{bookmark}}{\usepackage{hyperref}}
\hypersetup{
  pdftitle={Reading Assignments Sheet},
  pdfauthor={EES 3310/5310 Global Climate Change},
  hidelinks,
  pdfcreator={LaTeX via pandoc}}
\urlstyle{same} % disable monospaced font for URLs
\usepackage[margin=1in]{geometry}
\usepackage{graphicx,grffile}
\makeatletter
\def\maxwidth{\ifdim\Gin@nat@width>\linewidth\linewidth\else\Gin@nat@width\fi}
\def\maxheight{\ifdim\Gin@nat@height>\textheight\textheight\else\Gin@nat@height\fi}
\makeatother
% Scale images if necessary, so that they will not overflow the page
% margins by default, and it is still possible to overwrite the defaults
% using explicit options in \includegraphics[width, height, ...]{}
\setkeys{Gin}{width=\maxwidth,height=\maxheight,keepaspectratio}
% Set default figure placement to htbp
\makeatletter
\def\fps@figure{htbp}
\makeatother
\setlength{\emergencystretch}{3em} % prevent overfull lines
\providecommand{\tightlist}{%
  \setlength{\itemsep}{0pt}\setlength{\parskip}{0pt}}
\setcounter{secnumdepth}{-\maxdimen} % remove section numbering
\usepackage{jglucida}
\usepackage{float}
\usepackage{booktabs}
\usepackage[version=4]{mhchem}
\usepackage{siunitx}

\newcommand{\Ntwo}{\ce{N2}}
\newcommand{\Otwo}{\ce{O2}}
\newcommand{\Htwo}{\ce{H2}}
\newcommand{\COO}{\ce{CO2}}
\newcommand{\methane}{\ce{CH4}}

\newcommand{\water}{\ce{H2O}}
\newcommand{\silica}{\ce{SiO2}}
\newcommand{\calcite}{\ce{CaCO3}}
\newcommand{\CaSi}{\ce{CaSiO3}}
\newcommand{\carbonic}{\ce{H2CO3}}
\newcommand{\carbonicacid}{\carbonic}

\newcommand{\Hplus}{\ce{H+}}
\newcommand{\OH}{\ce{OH-}}
\newcommand{\Caplus}{\ce{Ca^2+}}
\newcommand{\Siplus}{\ce{Si^2+}}
\newcommand{\carb}{\ce{CO3^2-}}
\newcommand{\bicarb}{\ce{HCO3-}}
\newcommand{\bicarbonate}{\bicarb}
\newcommand{\carbonate}{\carb}
\newcommand{\silicate}{\ce{SiO3^2-}}

\newcommand{\degC}{\ensuremath{^\circ \mathrm{C}}}
\newcommand{\degF}{\ensuremath{^\circ \mathrm{F}}}
\newcommand{\pH}{p\ce{H}}
\newcommand{\permil}{\permille}

\newcommand{\Osixteen}{\ce{^{16}O}}
\newcommand{\Oeightteen}{\ce{^{18}O}}
\newcommand{\Ctwelve}{\ce{^{12}C}}
\newcommand{\Cthirteen}{\ce{^{13}C}}
\newcommand{\Cfourteen}{\ce{^{14}C}}
\newcommand{\Hone}{\ce{^1H}}
\newcommand{\deuterium}{\ce{^2H}}
\newcommand{\tritium}{\ce{^3H}}

\newcommand{\SLUGULATOR}{\textsc{slugulator}}
\newcommand{\GEOCARB}{\textsc{geocarb}}
\newcommand{\MODTRAN}{\textsc{modtran}}

\title{Reading Assignments Sheet}
\author{EES 3310/5310 Global Climate Change}
\date{Spring Semester, 2020}

\begin{document}
\maketitle

\hypertarget{disclaimer}{%
\section{Disclaimer}\label{disclaimer}}

This is a schedule of reading assignments through the entire term. I
have worked hard to plan the semester, but I may need to deviate from
this schedule, either because I decide that it's important to spend more
time on some subjects, or because new developments in either climate
science or climate policy require us to depart from my plans to discuss
current events and breaking news.

The most up-to-date versions of the reading assignments will be posted
on the \href{https://ees3310.jgilligan.org}{course web site}:
\url{https://ees3310.jgilligan.org}

\hypertarget{general-instructions-for-reading-assignments}{%
\section{General instructions for reading
assignments:}\label{general-instructions-for-reading-assignments}}

\begin{itemize}
\item
  Do the assigned reading \emph{before} you come to class on the date
  for which it is assigned. If you have questions or find the ideas
  presented in the readings confusing, I encourage you to ask questions
  in class.
\item
  Questions in the ``Reading Notes'' sections of the assignments are for
  you to think about to make sure you understand the material, but you
  do not have to write up your answers or turn them in. On tests, you
  are responsible for all the assigned readings, but topics I have
  highlighted in the reading notes are particularly important.
\item
  In addition to the questions I ask in the reading notes, look over the
  ``study questions'' at the end of each chapter in \emph{Understanding
  the Forecast} to check whether you understand the key facts and
  concepts from the chapter.

  Don't get confused between the \textbf{Study Questions} and the
  \textbf{Exercises} at the end of the chapter: Study Questions are for
  your own use in reviewing whether you understand the chapter, and the
  answers generally appear in the text, so if you don't know the answer,
  look back at the chapter. Several of the laboratory assignments ask
  you to do the Exercises, which are more challenging and ask you to
  apply the concepts from the chapter.
\end{itemize}

\hypertarget{mon.-jan.-6-introduction}{%
\section{Mon., Jan.~6: Introduction}\label{mon.-jan.-6-introduction}}

\hypertarget{reading}{%
\subsection{Reading:}\label{reading}}

No reading for today.

\hypertarget{wed.-jan.-8-what-is-climate-change}{%
\section{Wed., Jan.~8: What is Climate
Change?}\label{wed.-jan.-8-what-is-climate-change}}

\hypertarget{reading-1}{%
\subsection{Reading:}\label{reading-1}}

\hypertarget{required-reading-everyone}{%
\subsubsection{Required Reading
(everyone):}\label{required-reading-everyone}}

\begin{itemize}
\tightlist
\item
  Understanding the Forecast, Ch. 1.
\item
  Climate Casino, Ch. 2--3.
\end{itemize}

\hypertarget{reading-notes}{%
\subsubsection{Reading Notes:}\label{reading-notes}}

Read these chapters lightly. Do not try to memorize all the facts or
numbers, but try to get a feel for the way the authors write about
climate and what they think is important.

For Wednesday, come to class prepared to discuss the questions I list
for each book. (you don't need to write up answers to hand, but I want
you to be ready to discuss them):

\emph{Understanding the Forecast} is written by David Archer, a
prominent climate scientist. As you read it, try to get a sense for four
aspects in particular:

\begin{enumerate}
\def\labelenumi{\arabic{enumi}.}
\tightlist
\item
  What kinds of things (both human and natural) cause the earth's
  climate to change?
\item
  Why are carbon and energy so important?
\item
  Very roughly, how much has the earth's temperature changed in the past
  and how much do we expect it to change in the next few centuries?
\item
  Should we worry about climate change and human activities that cause
  it?
\end{enumerate}

\hypertarget{questions-to-think-about}{%
\subsubsection{Questions to think
about:}\label{questions-to-think-about}}

\begin{enumerate}
\def\labelenumi{\arabic{enumi}.}
\tightlist
\item
  What is climate, and how is it different from weather?
\item
  What determines the temperature of the earth, and what are some things
  that can cause the temperature to change?
\item
  What is global warming? What kinds of risks do we worry about in
  connection with it?
\item
  What are several reasons why we emphasize carbon dioxide (\COO{}) when
  we talk about global warming?
\item
  About how much has earth's temperature varied in the past thousand
  years? In the last 25,000 years? How much do we expect it to change in
  the next hundred years?
\item
  What are some extreme climatic or weather events that we've seen
  recently around the world (think of things in the assigned readings,
  but also other things you've experienced or heard about in the news).
\item
  What kinds of trends do we see in natural disasters?
\item
  How would you know whether or not the kinds of events and trends in
  your answers to the two previous questions are caused by human
  interference with climate?
\item
  What possible responses can we (the population of the earth) take to
  respond to global warming?
\end{enumerate}

\emph{Climate Casino} is written by William Nordhaus, a prominent
economist. He has a different emphasis than Archer does. Read Chapter 1
more carefully, and mostly skim Chapter 2 lightly, but do pay attention
to the section on p.~15 where he asks, ``why read a book about climate
change by an economist?'' Chapter 1 presents an outline of the book. As
you read it, pay attention to the way Nordhaus connects the science and
economics of climate change. The figure on p.~10 illustrates these
connections and it is important to understand what he is saying with it.

At the end of the chapter, Nordhaus presents three things he thinks
people around the planet \textbf{must} do about climate change. Later in
the semester we will read a book that disagrees vehemently with most of
Nordhaus's analysis and recommendations. For now, ask yourself whether
you agree with Nordhaus's three points, why you agree or disagree, and
whether he has made a strong case for these points in this chapter.

\hypertarget{questions-to-think-about-1}{%
\subsubsection{Questions to think
about}\label{questions-to-think-about-1}}

\begin{enumerate}
\def\labelenumi{\arabic{enumi}.}
\tightlist
\item
  Why does Nordhaus emphasize the difference between \textbf{managed},
  \textbf{unmanaged}, and \textbf{unmanageable} human and natural
  systems?
\item
  What does Nordhaus mean by \textbf{tipping points} in the earth's
  climate system? Why are they important?
\item
  What does Nordhaus mean by \textbf{mitigation} of climate change?
  (p.~6)
\item
  Why does Nordhaus think that ``the economics of climate change is
  straightforward?'' What policy does his ``straightforward'' analysis
  recommmend?
\item
  What does Nordhaus mean when he calls emission of \COO{} into the
  atmosphere an \textbf{externality}? (p.~6) Why is this important for
  policymakers and for his own preferred policy?
\end{enumerate}

Also come to class ready to ask questions about parts of the chapter
that you didn't understand as well as you'd like or that you found
unconvincing, or things you just want to know. If you email me questions
by Tuesday evening, I will try to address them during class Tuesday.

\hypertarget{fri.-jan.-10-energy-balance-and-climate}{%
\section{Fri., Jan.~10: Energy Balance and
Climate}\label{fri.-jan.-10-energy-balance-and-climate}}

\hypertarget{reading-2}{%
\subsection{Reading:}\label{reading-2}}

\hypertarget{required-reading-everyone-1}{%
\subsubsection{Required Reading
(everyone):}\label{required-reading-everyone-1}}

\begin{itemize}
\tightlist
\item
  Understanding the Forecast, Ch. 2--3, pp.~9--23.
\end{itemize}

\hypertarget{reading-notes-1}{%
\subsubsection{Reading Notes:}\label{reading-notes-1}}

Focus on pp.~13--23 in \emph{Understanding the Forecast}. You need to
understand the calculations of the ``bare-rock'' model on pp.~19--23.
The intermediate steps are not as important as two equations: \[
  F_{\text{out}} = F_{\text{in}}\quad\text{at equilibrium,}
\] and equation (3.1), which describes the bare-rock model: \[
  T_{\text{earth}} =
  \sqrt[4]{\frac{(1-\alpha)I_{\text{in}}}{4\epsilon\sigma}}
\] (Helpful hint: to take a fourth root easily with your calculator,
just press the square root key twice.)

Questions to think about (\textbf{not} to write up and turn in):

\begin{itemize}
\item
  What is blackbody radiation? What is a ``blackbody'' anyway?
\item
  Why is it that the sun gives off visible light, but the earth does
  not?
\item
  When the earth absorbs energy from sunlight, where does the energy go
  initially? Where is the final destination of that energy?
\item
  What is the Stefan-Boltzmann equation, and why is it important?
\item
  What does the Stefan-Boltzmann equation tell us would happen if the
  sun got hotter? What would happen if the Earth got hotter?
\item
  Study table 3.1 on p.~23 of \emph{Understanding the Forecast} (ignore
  the column ``\(T_{\text{1 layer}}\)'' because we don't get to that
  until later in the chapter.):

  \begin{itemize}
  \tightlist
  \item
    Why is the sunlight brighter on Venus than on Earth, and dimmer on
    Mars?
  \item
    Why is the ``bare-rock'' temperature of Venus lower than Earth, even
    though it gets more sunlight?
  \item
    Why do you suppose the actual observed temperature at the surface of
    Venus is so much hotter than the ``bare rock'' temperature?
  \end{itemize}
\item
  At the top of p.~20, why does Archer write,
  \(F_{\text{out}} = F_{\text{in}}\)? What would happen if
  \(F_{\text{out}} \ne F_{\text{in}}\)
\item
  Without getting bogged down in the details of the numbers, why are the
  areas used to calculate the incoming and outgoing energy fluxes
  different? (Figures 3.1 and 3.2 explain this)
\item
  If the sun got 5\% brighter, approximately how many degrees warmer
  would the earth become?
\end{itemize}

\hypertarget{mon.-jan.-13-greenhouse-effect}{%
\section{Mon., Jan.~13: Greenhouse
Effect}\label{mon.-jan.-13-greenhouse-effect}}

\hypertarget{reading-3}{%
\subsection{Reading:}\label{reading-3}}

\hypertarget{required-reading-everyone-2}{%
\subsubsection{Required Reading
(everyone):}\label{required-reading-everyone-2}}

\begin{itemize}
\tightlist
\item
  Understanding the Forecast, Ch. 3, pp.~23--26.
\end{itemize}

\hypertarget{reading-notes-2}{%
\subsubsection{Reading Notes:}\label{reading-notes-2}}

Study the one-layer model. We will work through it in detail during
class.

\hypertarget{wed.-jan.-15-greenhouse-gases}{%
\section{Wed., Jan.~15: Greenhouse
Gases}\label{wed.-jan.-15-greenhouse-gases}}

\hypertarget{reading-4}{%
\subsection{Reading:}\label{reading-4}}

\hypertarget{required-reading-everyone-3}{%
\subsubsection{Required Reading
(everyone):}\label{required-reading-everyone-3}}

\begin{itemize}
\tightlist
\item
  Understanding the Forecast, Ch. 4.
\end{itemize}

\hypertarget{reading-notes-3}{%
\subsubsection{Reading Notes:}\label{reading-notes-3}}

Undergraduates need only understand band saturation qualitatively (i.e.,
you don't need to worry about equations 4.1--4.3). Graduate students do
need to know and be able to apply these equations.

You should understand why molecules with many atoms are more powerful
greenhouse gases than molecules with fewer atoms. Can a single atom
(e.g., noble gases such as helium) act as greenhouse gases? What about
nitrogen (\Ntwo{}) and oxygen (\Otwo{}) molecules, which make up almost
99\% of the atmosphere?

What is the ``atmospheric window,'' that spans the range of about
800--1250 \(\text{cycles}/\text{cm}\) (wavelengths of 8--12 \(\mu\)m)?

\hypertarget{understanding-emissions-spectra-seen-by-satellites}{%
\subsubsection{Understanding emissions spectra seen by
satellites}\label{understanding-emissions-spectra-seen-by-satellites}}

\textbf{This is important:} The key to this section is understanding
diagrams, such as Fig. 4.3 and 4.5. These diagrams show the brightness
of infrared light emitted at different frequencies. The smooth lines
correspond to the radiation a perfect blackbody would give off at
different temperatures. Hotter bodies give off more energy and cooler
bodies give off less.

The jagged line is the actual radiation given off by the atmosphere, as
seen by a satellite in space. Because the atmosphere is cooler higher
up, emissions that follow a cooler blackbody curve for some part of the
spectrum mean the infrared at those wavenumbers is emitted by greenhouse
gases higher in the atmosphere. Brighter emissions that follow a hotter
blackbody curve for some part of the spectrum means hotter temperatures,
and hence emission at those wavenumbers comes from gases closer to the
ground.

As you add more of a gas, it absorbs more infrared light, so the
emissions from the ground or from lower in the atmosphere can't get
through---they're absorbed by higher layers of the atmosphere---and
thus, the only emissions that get out to space come from higher (and
thus colder) layers. This is why you see lower emissions (colder
temperatures) for the wavenumbers at which powerful greenhouse gases,
such as \COO{} and ozone, absorb.

Here are some review questions to check whether you understand this
material well:

\begin{itemize}
\tightlist
\item
  Water vapor is a powerful greenhouse gas. Why do you suppose we don't
  see extremely cold temperatures for the part of the spectrum (100--600
  \(\text{cycles}/\text{cm}\)) where \emph{it} absorbs?
\item
  If greenhouse gases make earth warmer, why does adding \COO{} in make
  the emissions temperatures become smaller in Fig. 4-5?
\end{itemize}

\hypertarget{band-saturation}{%
\subsubsection{Band saturation}\label{band-saturation}}

\textbf{This is important for understanding why some greenhouse gases
are more powerful than others} What is band saturation? Why does
increasing \COO{} from 10 to 100 parts per million or from 100 to 1000
parts per million have much less effect than increasing it from 0 to 10
parts per million? Think by analogy. Suppose you smear a tablespoon of
black ink all over a white piece of paper. How much will that affect its
brightness? Now suppose you smear four more tablespoons of ink on the
stained paper. Which has the bigger effect on the whiteness of the
paper: the first spill of ink, or the second (much bigger) one?

Now, looking at the atmosphere with the current amount of greenhouse
gases, can you speculate why adding a million molecules of methane to
the atmosphere will produce 20 times more warming than adding a million
molecules of \COO{}?

\hypertarget{fri.-jan.-17-vertical-structure-of-the-atmosphere}{%
\section{Fri., Jan.~17: Vertical Structure of the
Atmosphere}\label{fri.-jan.-17-vertical-structure-of-the-atmosphere}}

\hypertarget{reading-5}{%
\subsection{Reading:}\label{reading-5}}

\hypertarget{required-reading-everyone-4}{%
\subsubsection{Required Reading
(everyone):}\label{required-reading-everyone-4}}

\begin{itemize}
\tightlist
\item
  Understanding the Forecast, Ch. 5.
\end{itemize}

\hypertarget{reading-notes-4}{%
\subsubsection{Reading Notes:}\label{reading-notes-4}}

The key thing in this chapter is understanding why the troposphere gets
colder as you go higher, and how this phenomenon (described by the term
\textbf{lapse rate}), contributes to the greenhouse effect.

First, the chapter starts by defining a \textbf{skin layer}, which is an
important concept, and connects it to the greenhouse effect.

Next, it moves on to describe how air pressure varies with altitude. I
find the discussion on pp.~46--49 more confusing than it needs to be.
Don't get bogged down in the details of \emph{why} the equation at the
bottom of p.~48 is true. Just understand \emph{what} that equation tells
you: \[
  P(z) = 1~\text{atm}\cdot e^{-z/8~\text{km}}
\] You might find it easier and more intuitive to work with another way
to write the same equation: \[
  P(z) = 1~\text{atm}\cdot 2^{-z/5.5~\text{km}}
\] If we think of it this way, it's intuitive that every 5.5 km you go
up in the atmosphere, the pressure drops by half: at 5.5 km, it's 50\%
what it is at sea level. At 11.0 km it's 25\%; at 16.5 km it's 12.5\%,
and so on.

Third, we learn why air cools off as it rises through the atmosphere.
It's because it expands as the pressure surrounding it drops. Similarly,
as air descends to lower altitudes, it compresses and gets hotter.

Fourth, we learn about latent heat from water evaporating and
condensing. Latent heat is responsible for a large fraction of the heat
transport around the atmosphere. Latent heat is also very important
because it affects lapse rate. How does the lapse rate in
\textbf{saturated} air (i.e., air with 100\% relative humidity) differ
from the lapse rate in \textbf{unsaturated} air (with relative humidity
\(<100%
\))? Why? (Hint: the relavant material appears in the section on
\textbf{moist convection}, not the section on latent heat.)

Fifth, we learn about convection: as you heat air it tends to rise and
this moves heat around the atmosphere. The key concept regarding
convection is \textbf{stability} vs. \textbf{instability}. Unstable air
undergoes convection (as the troposphere does), whereas stable air does
not, and instead remains \textbf{stratified}, as the stratosphere does.

Sixth (and finally), we put everything together to see the connection
between lapse rates and the greenhouse effect. The key result is that
adding greenhouse gases raises the height of the ``skin'' and this added
height, together with the lapse rate, tells us how much the surface
temperature will warm up. The book summarizes these five points
concisely on p.~55.

\hypertarget{mon.-jan.-20-martin-luther-king-jr.-day}{%
\section{Mon., Jan.~20: Martin Luther King,
Jr.~Day}\label{mon.-jan.-20-martin-luther-king-jr.-day}}

Martin Luther King, Jr.~Day, no class.

\hypertarget{wed.-jan.-22-review-of-greenhouse-effect}{%
\section{Wed., Jan.~22: Review of Greenhouse
Effect}\label{wed.-jan.-22-review-of-greenhouse-effect}}

\hypertarget{reading-6}{%
\subsection{Reading:}\label{reading-6}}

No reading for today.

\hypertarget{notes}{%
\subsubsection{Notes:}\label{notes}}

As you prepare for class, think about:

\begin{itemize}
\tightlist
\item
  What makes some greenhouse gases more powerful than others?
\item
  What are the important differences between the layer-model approach to
  calculating temperature and the skin-model on pp.~45--46?
\item
  How does band saturation work? How can you recognize saturation in
  \MODTRAN{}?
\item
  How do you use \MODTRAN{} to calculate the effect of doubling \COO{}?
\item
  How do you use the full-spectrum model to calculate the effect of
  doubling \COO{}?
\end{itemize}

\hypertarget{fri.-jan.-24-feedbacks}{%
\section{Fri., Jan.~24: Feedbacks}\label{fri.-jan.-24-feedbacks}}

\hypertarget{reading-7}{%
\subsection{Reading:}\label{reading-7}}

\hypertarget{required-reading-everyone-5}{%
\subsubsection{Required Reading
(everyone):}\label{required-reading-everyone-5}}

\begin{itemize}
\tightlist
\item
  Understanding the Forecast, Ch. 7, pp.~73--81.
\end{itemize}

\hypertarget{reading-notes-5}{%
\subsubsection{Reading Notes:}\label{reading-notes-5}}

\begin{itemize}
\item
  Understand the concept of positive and negative feedbacks.
\item
  Understand how the following feedbacks work and whether each is
  positive or negative:

  \begin{itemize}
  \tightlist
  \item
    Stefan-Boltzmann feedback
  \item
    Ice-albedo feedback
  \item
    Water vapor feedback
  \end{itemize}
\item
  What is a \textbf{runaway greenhouse effect}? What prevents Earth from
  having a runaway greenhouse?
\end{itemize}

\hypertarget{mon.-jan.-27-ocean-and-biosphere-feedbacks}{%
\section{Mon., Jan.~27: Ocean and Biosphere
Feedbacks}\label{mon.-jan.-27-ocean-and-biosphere-feedbacks}}

\hypertarget{reading-8}{%
\subsection{Reading:}\label{reading-8}}

\hypertarget{required-reading-everyone-6}{%
\subsubsection{Required Reading
(everyone):}\label{required-reading-everyone-6}}

\begin{itemize}
\tightlist
\item
  Understanding the Forecast, Ch. 7, pp.~81--84.
\item
  Handout: \href{/files/reading_handouts/Feedback_Handout.pdf}{Jonathan
  Gilligan, ``Handout on Feedbacks''}.
\end{itemize}

\hypertarget{reading-notes-6}{%
\subsubsection{Reading Notes:}\label{reading-notes-6}}

\begin{itemize}
\item
  Why are cloud feedbacks more complicated and uncertain than other
  feedbacks?
\item
  What is El Niño and how do feedbacks in the ocean-atmosphere system
  create the El Niño/La Niña cycle? How does this cycle affect the
  global climate?
\item
  How do feedbacks between climate and the biosphere work?
\item
  In the handout, pay close attenton to two different ways of talking
  about feedbacks: If we \textbf{force} the climate by changing the
  brightness of the sun, the bare-rock temperature of the earth adjusts
  until the average flux of outgoing radiation equals the average flux
  of incoming radiation. Sometimes we count this adjustment as a
  negative feedback (called the \textbf{Stefan-Boltzmann feedback}) that
  maintains radiative balance. Other times we consider it to be a simple
  response to the changing amount of radiation.

  The handout calls this adjustment the Stefan-Boltzmann feedback
  (\(f_0\)) and calculates the other feedbacks (water-vapor, lapse-rate,
  ice-albedo, etc.) as additional feedbacks. This lets us write a
  straightforward equation for the effect of all the different feedbacks
  on the earth's temperature, but it is important to be clear that when
  most people (including most scientists) talk about feedbacks in the
  climate system, they do not include the Stefan-Boltzmann feedback.
\end{itemize}

\hypertarget{wed.-jan.-29-the-carbon-cycle-ocean-and-biosphere}{%
\section{Wed., Jan.~29: The Carbon Cycle: Ocean and
Biosphere}\label{wed.-jan.-29-the-carbon-cycle-ocean-and-biosphere}}

\hypertarget{reading-9}{%
\subsection{Reading:}\label{reading-9}}

\hypertarget{required-reading-everyone-7}{%
\subsubsection{Required Reading
(everyone):}\label{required-reading-everyone-7}}

\begin{itemize}
\tightlist
\item
  Understanding the Forecast, Ch. 8, pp.~89--97.
\end{itemize}

\hypertarget{reading-notes-7}{%
\subsubsection{Reading Notes:}\label{reading-notes-7}}

\begin{itemize}
\tightlist
\item
  Know the different forms carbon takes in the earth system: \COO{} and
  organic gases in the atmosphere; dissolved inorganic carbon in the
  hydrosphere (carbonic acid and related ions: \carbonicacid{},
  \bicarbonate{}, \carbonate{}); solid organic and inorganic matter in
  the lithosphere (what's the difference between organic and inorganic?
  What are dominant forms that each takes in the lithosphere?); and
  living organic carbon in the biosphere.
\item
  What are \textbf{oxidation} and \textbf{reduction} and how do they
  affect organic and inorganic carbon?
\item
  Focus intently on Fig. 8.1: get a feel for the size of each reservoir
  and the magnitude of flux between the different reservoirs. Don't feel
  that you have to memorize the numbers, but you should have a good feel
  for which are larger, which are smaller, and a general sense of the
  range of sizes.
\item
  What are the dominant mechanisms by which carbon moves from one
  reservoir to another? Which processes are fast and which are slow?
\item
  In Fig. 8-2, why are the annual wiggles in the atmospheric carbon
  concentration so much bigger in Hawaii than in New Zealand?
\item
  How might feedbacks in the carbon cycle destabilize the global
  climate?
\item
  Try to get a rough feel for the orbital forcing of climate, but don't
  stress about the details. We'll dig into this in much more depth when
  we look at climates of the past, on Feb.~5--7. The key here is to
  understand that small variations in the earth's orbit lead to small
  forcings on the climate, which are dramatically amplified by a
  positive feedback in the carbon cycle to produce the cycle of ice ages
  that the earth experienced over the past 2 million years or so.
\end{itemize}

\hypertarget{fri.-jan.-31-the-carbon-cycle-mineral-weathering}{%
\section{Fri., Jan.~31: The Carbon Cycle: Mineral
Weathering}\label{fri.-jan.-31-the-carbon-cycle-mineral-weathering}}

\hypertarget{reading-10}{%
\subsection{Reading:}\label{reading-10}}

\hypertarget{required-reading-everyone-8}{%
\subsubsection{Required Reading
(everyone):}\label{required-reading-everyone-8}}

\begin{itemize}
\tightlist
\item
  Understanding the Forecast, Ch. 8, pp.~95--101.
\end{itemize}

\hypertarget{reading-notes-8}{%
\subsubsection{Reading Notes:}\label{reading-notes-8}}

\begin{itemize}
\item
  Try to get a rough feel for the orbital forcing of climate, but don't
  stress about the details. We'll dig into this in much more depth when
  we look at climates of the past, on Feb.~5--7. The key here is to
  understand that small variations in the earth's orbit lead to small
  forcings on the climate, which are dramatically amplified by a
  positive feedback in the carbon cycle to produce the cycle of ice ages
  that the earth experienced over the past 2 million years or so.
\item
  Understand the feedback in the sedimentary rock cycle. This is very
  important, but very slow. It acts as a thermostat for the earth, but
  takes millions of years to act, so although it will ultimately fix any
  global warming that people cause, it will do so much too slowly to
  protect our civilization from the effects of climate change.

  The key to the sedimentary rock part of the carbon cycle is the
  transformation of \textbf{silicate minerals} to \textbf{carbonate
  minerals} through weathering. You should understand how this cycle
  works, how it changes in response to changing climate in order to act
  as a thermostat (negative feedback), and why this cycle works well on
  Earth but not on Mars or Venus. One clue is the graph in Fig. 7.2 on
  p.~76.
\end{itemize}

\hypertarget{mon.-feb.-3-perturbing-the-carbon-cycle}{%
\section{Mon., Feb.~3: Perturbing the Carbon
Cycle}\label{mon.-feb.-3-perturbing-the-carbon-cycle}}

\hypertarget{reading-11}{%
\subsection{Reading:}\label{reading-11}}

\hypertarget{required-reading-everyone-9}{%
\subsubsection{Required Reading
(everyone):}\label{required-reading-everyone-9}}

\begin{itemize}
\tightlist
\item
  Understanding the Forecast, Ch. 10.
\end{itemize}

\hypertarget{reading-notes-9}{%
\subsubsection{Reading Notes:}\label{reading-notes-9}}

Some key points to be sure you understand:

\begin{itemize}
\tightlist
\item
  How is the steady-state concentration of methane in the atmosphere
  related to the rate of methane emissions?
\item
  What are the largest natural and anthropogenic sources of methane? How
  does anthropogenic methane emission compare to natural emission?
\item
  What are the dominant anthropogenic sources of \COO{} emissions?
\item
  What do we know about the land as a sink for carbon?
\item
  What is \COO{} fertilization and how might it affect the role of the
  land as a carbon sink? What are the potential benefits to plants of
  higher \COO{} concentrations and what constraints are there on the
  extent to which these benefits might be realized in practice?
\item
  Understand the ocean carbon sink. Where, specifically, does the ocean
  work effectively to remove \COO{} from the atmosphere and how might
  this sink be affected by rising \COO{} levels?
\item
  What is \textbf{ocean ventilation}? Why is it important and how fast
  does it work?
\item
  What is the \textbf{thermocline} and why is it important to the ocean
  carbon sink?
\item
  Why would increasing carbon dioxide in the atmosphere make the oceans
  become more acidic? How confident can we be of predictions about ocean
  acidification?
\item
  What is \textbf{buffering} and how would it affect ocean
  acidification?
\item
  What is the biological pump in the oceans? Are there things people
  might do to speed up the biological pump
\item
  In the big picture, consider how these pieces fit together: how might
  rising temperatures and increasing drought around the world (two
  things scientists are very confident will happen as a result of rising
  greenhouse gas concentrations) affect the various carbon sources and
  sinks? How might ocean acidification affect the biological pump?
\end{itemize}

\hypertarget{wed.-feb.-5-climates-of-the-past}{%
\section{Wed., Feb.~5: Climates of the
Past}\label{wed.-feb.-5-climates-of-the-past}}

\hypertarget{reading-12}{%
\subsection{Reading:}\label{reading-12}}

\hypertarget{required-reading-everyone-10}{%
\subsubsection{Required Reading
(everyone):}\label{required-reading-everyone-10}}

\begin{itemize}
\tightlist
\item
  Understanding the Forecast, Ch. 11, pp.~135--145.
\end{itemize}

\hypertarget{reading-notes-10}{%
\subsubsection{Reading Notes:}\label{reading-notes-10}}

\begin{itemize}
\item
  The last hundred years: Focus on two things in particular:

  \begin{enumerate}
  \def\labelenumi{\arabic{enumi}.}
  \tightlist
  \item
    What is the evidence that the planet has warmed significantly in the
    last century?
  \item
    What is the evidence that human emissions of greenhouse gases are
    responsible?
  \end{enumerate}
\item
  The last thousand years: This gets into some very tricky and
  controversial material that was central to the infamous
  ``climategate'' scandal in 2009. I will not spend a huge amount of
  time on this, but I will attempt to explain in class what the
  controversy was about and why it isn't nearly as big a deal as the
  media have made it out to be.

  Some things to understand about the past millennium are:

  \begin{itemize}
  \tightlist
  \item
    What is a \textbf{proxy} for ancient temperatures? What are examples
    of commonly used proxies?
  \item
    What problems do scientists face trying to get past temperatures
    from proxies?
  \item
    What were the \textbf{Medieval Warm Period} and the \textbf{Little
    Ice Age}? Roughly when did they happen? What are the big questions
    scientists have about them?
  \end{itemize}
\end{itemize}

\hypertarget{fri.-feb.-7-the-pleistocene-ice-ages}{%
\section{Fri., Feb.~7: The Pleistocene Ice
Ages}\label{fri.-feb.-7-the-pleistocene-ice-ages}}

\hypertarget{reading-13}{%
\subsection{Reading:}\label{reading-13}}

\hypertarget{required-reading-everyone-11}{%
\subsubsection{Required Reading
(everyone):}\label{required-reading-everyone-11}}

\begin{itemize}
\tightlist
\item
  Understanding the Forecast, Ch. 7, p.~84. (You already read this, but
  please review it again).
\item
  Understanding the Forecast, Ch. 8, pp.~93--97. (You already read this,
  but please review it again).
\item
  Understanding the Forecast, Ch. 11, pp.~147--149.
\item
  Handout: \href{/files/reading_handouts/Isotope_Handout.pdf}{Jonathan
  Gilligan, ``Handout on Isotopes,''}.
\end{itemize}

\hypertarget{reading-notes-11}{%
\subsubsection{Reading Notes:}\label{reading-notes-11}}

Focus on the Pleistocene Ice Ages, which extended from roughly 2.75
million years ago (different scientists attribute different dates to the
exact beginning of the Pleistocene) to about 10,000 years ago when our
current warm period, the Holocene, began when the glaciers retreated for
the last time.

We have two principal questions about the Pleistocene: How do we know
what climate was like then? And how do we know what caused the cycle of
ice ages?

\begin{itemize}
\tightlist
\item
  Pay particular attention to the way oxygen isotopes in samples of ice
  can be used to tell the temperature hundreds of thousands of years ago
  and the way oxygen isotopes in sediments at the bottom of the ocean
  can tell us about the amount of ice in the glaciers around the world,
  but both are useful.
\item
  Recall the discussion of carbon dioxide feedbacks on p.~84. This is
  crucial.
\item
  What kinds of instabilities did we see in the climates of the
  Pleistocene? Are those instabilities relevant to questions of how the
  climate will change as we increase the atmospheric greenhouse gas
  levels over the coming century?
\end{itemize}

\hypertarget{mon.-feb.-10-review}{%
\section{Mon., Feb.~10: Review}\label{mon.-feb.-10-review}}

\hypertarget{reading-14}{%
\subsection{Reading:}\label{reading-14}}

No reading for today.

\hypertarget{notes-1}{%
\subsubsection{Notes:}\label{notes-1}}

Take time to review the reading from the last week (Chapters 10 and 11
and the handout on isotopes). Come to class with questions about the
carbon cycle and past climates.

\hypertarget{wed.-feb.-12-climate-models}{%
\section{Wed., Feb.~12: Climate
Models}\label{wed.-feb.-12-climate-models}}

\hypertarget{reading-15}{%
\subsection{Reading:}\label{reading-15}}

\hypertarget{required-reading-everyone-12}{%
\subsubsection{Required Reading
(everyone):}\label{required-reading-everyone-12}}

\begin{itemize}
\tightlist
\item
  Climate Casino, Ch. 3--4.
\end{itemize}

\hypertarget{reading-notes-12}{%
\subsubsection{Reading Notes:}\label{reading-notes-12}}

Chapter 3 treats global warming as a scientific consequence of an
economic problem. As you read, consider the following questions. Some
have clear answers while others are beyond the knowledge of experts and
I ask them to challenge you to think about hard problems. On p.~30,
Nordhaus describes three components that drive \COO{} emissions:
population, per-capita GDP, and the carbon intensity of the economy. A
mathematical expression for this relationship is known as the
\textbf{Kaya identity}, and we will study this in great depth on
Mar.~9--13 and in the lab project on decarbonizing the energy supply.

\begin{itemize}
\tightlist
\item
  Why don't free markets manage greenhouse gas emissions well?
\item
  Are \COO{} emissions going up or down? Why?
\item
  What is carbon intensity? Is it going up or down in the US? Why?
\item
  When Nordhaus writes about models for predicting future climate
  change, he distinguishes between \textbf{predictions} and
  \textbf{projections}. What is the difference and why is it important?
\item
  Table 1 on p.~31 shows two projections of future \COO{} emissions. Why
  are they different? How can we tell which is a better prediction?
\item
  What is an \textbf{integrated assessment model} (\textbf{IAM})?
\item
  What are the biggest sources of uncertainty in predicting future
  \COO{} emissions?
\item
  How much can we trust models of future climate change? What should we
  consider when deciding how much to trust a model?
\end{itemize}

Chapter 4 looks at what we do and don't know about future climate
change.

\begin{itemize}
\tightlist
\item
  Pages 37--42 are largely a review of material we studied in much
  greater depth in the first few weeks of the term. You can read through
  it quickly.
\item
  What is the difference between \textbf{transient} and
  \textbf{equilibrium} response to \COO{} emissions? What would we
  expect to happen to the global temperature if everyone around the
  world completely stopped burning fossil fuels this afternoon?
\item
  Figure 9 shows several different projections for how temperature might
  change over the rest of this century. What is the biggest reason the
  projections don't all agree with each other?
\item
  Pay close attention to the bullet points on pp.~47--48.
\item
  How does Nordhaus recommend that we think about the uncertainties in
  predictions about the climate?
\item
  Given these uncertainties, can we trust climate models and can they be
  useful?
\end{itemize}

\hypertarget{fri.-feb.-14-future-climate-change}{%
\section{Fri., Feb.~14: Future Climate
Change}\label{fri.-feb.-14-future-climate-change}}

\hypertarget{reading-16}{%
\subsection{Reading:}\label{reading-16}}

\hypertarget{required-reading-everyone-13}{%
\subsubsection{Required Reading
(everyone):}\label{required-reading-everyone-13}}

\begin{itemize}
\tightlist
\item
  Climate Casino, Ch. 5.
\item
  Understanding the Forecast, Ch. 12, pp.~153--164.
\end{itemize}

\hypertarget{reading-notes-13}{%
\subsubsection{Reading Notes:}\label{reading-notes-13}}

One of the big worries with future climate change has to do with tipping
points. As you read Ch. 5 of \emph{The Climate Casino}, you should focus
on understanding what tipping points are and what some of the specific
tipping points are worry climate experts. Fig. 12 and the accompanying
discussion about melting ice sheets is a key example and you should
understand it. If we raise the temperature and a large fraction of the
Greenland ice sheet melts, will reducing the temperature cause it to
eventually grow back to its original size? Why or why not?

Chapter 12 of \emph{Understanding the Forecast} presents a lot more
detail about \textbf{abrupt climate change} (tipping points), melting
ice sheets, and sea-level rise. The earlier part of the chapter also
discusses the phenomenon of \textbf{global weirding}, in which the
weather becomes less and less predictable as the global temperature
rises. Indeed, recent scientific studies confirm predictions made years
ago that global warming would lead to both increased drought and
increased flooding. How is this possible?

\hypertarget{mon.-feb.-17-catching-up-and-review}{%
\section{Mon., Feb.~17: Catching up and
Review}\label{mon.-feb.-17-catching-up-and-review}}

\hypertarget{reading-17}{%
\subsection{Reading:}\label{reading-17}}

No reading for today.

\hypertarget{notes-2}{%
\subsubsection{Notes:}\label{notes-2}}

We will use today to catch up and review for the midterm exam. Bring
your questions.

\hypertarget{wed.-feb.-19-midterm-exam}{%
\section{Wed., Feb.~19: Midterm Exam}\label{wed.-feb.-19-midterm-exam}}

\hypertarget{reading-18}{%
\subsection{Reading:}\label{reading-18}}

No reading for today.

\hypertarget{notes-3}{%
\subsubsection{Notes:}\label{notes-3}}

The midterm exam will cover all the reading through Wednesday, Feb.~12.

I will provide a sheet with the important numbers (such as the
Stefan-Boltzmann constant, the lapse rate of the atmosphere, the solar
constant, and so forth), and the important equations (such as the
Stefan-Boltzmann law, the barometric law, and the chemical reactions
that control carbon-dioxide buffering in the oceans).

Remember to bring a calculator, \#2 pencils, and an eraser.

\hypertarget{fri.-feb.-21-uncertainty-about-future-climates}{%
\section{Fri., Feb.~21: Uncertainty about Future
Climates}\label{fri.-feb.-21-uncertainty-about-future-climates}}

\hypertarget{reading-19}{%
\subsection{Reading:}\label{reading-19}}

\hypertarget{required-reading-everyone-14}{%
\subsubsection{Required Reading
(everyone):}\label{required-reading-everyone-14}}

\begin{itemize}
\tightlist
\item
  Understanding the Forecast, Ch. 12, pp.~164--166.
\item
  Climate Casino, Ch. 24.
\item
  The Climate Fix, Ch. 1, pp.~1--24.
\end{itemize}

\hypertarget{reading-notes-14}{%
\subsubsection{Reading Notes:}\label{reading-notes-14}}

This is a good place for us to reflect on what we do and don't know
about the science of climate change. As we review the science we have
studied so far, think about the contrasting views Nordhaus and Pielke
present in \emph{Climate Casino} and \emph{The Climate Fix} about what
we do and don't know and whether people should be skeptical of
scientists about global warming. In particular, focus on:

\begin{itemize}
\tightlist
\item
  What is \textbf{scientific consensus}, is there consensus on global
  warming, and should we trust it? Where do Nordhaus and Pielke agree
  and where do they disagree about this?
\item
  How should we think about contrarians or skeptics who disagree with
  the consensus?
\item
  How should we think about scientific uncertainty? Where do Nordhaus
  and Pielke agree and where do they disagree about this?
\item
  Can we make policy when there is still disagreement and uncertainty
  about climate science? How do Nordhaus and Pielke feel about this?
\end{itemize}

\hypertarget{mon.-feb.-24-how-will-climate-change-affect-our-lives-part-1}{%
\section{Mon., Feb.~24: How Will Climate Change Affect Our Lives? (Part
1)}\label{mon.-feb.-24-how-will-climate-change-affect-our-lives-part-1}}

\hypertarget{reading-20}{%
\subsection{Reading:}\label{reading-20}}

\hypertarget{required-reading-everyone-15}{%
\subsubsection{Required Reading
(everyone):}\label{required-reading-everyone-15}}

\begin{itemize}
\tightlist
\item
  Climate Casino, Ch. 6--9.
\end{itemize}

\hypertarget{reading-notes-15}{%
\subsubsection{Reading Notes:}\label{reading-notes-15}}

This reading assignment covers many pages, but the material is
descriptive, not mathematical. What I am looking for you to get out of
this is a sense of the different kinds of impacts that climate change
might have on our lives and the lives of people around the planet.

\begin{itemize}
\item
  Pay particular attention to the distinction Nordhaus draws between
  \textbf{managed}, \textbf{unmanaged}, and \textbf{unmanageable}
  systems.
\item
  Also pay attention to the discussions of \textbf{adaptation} and
  \textbf{mitigation}.

  \begin{itemize}
  \item
    \textbf{Mitigating factors} mean aspects of climate change that may
    be beneficial and mitigate the damage caused by the harmful aspects.
    This can be confusing because in the context of climate policy
    \textbf{mitigation} usually means reducing the amount of climate
    change (e.g., by reducing greenhouse gas emissions), whereas
    \textbf{mitigating factors} are things that reduce the impact that a
    given amount of climate change will have on people's lives.
  \item
    \textbf{Adaptation} means changes people make in the way they live
    and the kinds of economic activities they pursue in order to adapt
    to living in a different climate.

    Adaptation and mitigating factors are important because they show us
    that there can be more to climate change policy than just reducing
    greenhouse gas emissions. Again, to point out how confusing this
    terminology can be, \emph{mitigating factors} (things that reduce
    the impact of climate change on people's lives) show that there is
    more to climate policy than \emph{mitigation} (reducing greenhouse
    gas emissions).
  \end{itemize}
\item
  In the three chapters on details (7--9), try to get a feel for the
  following questions:

  \begin{enumerate}
  \def\labelenumi{\arabic{enumi}.}
  \tightlist
  \item
    How severe are the threats likely to be to human well-being?
  \item
    Are certain groups of people especially vulnerable?
  \item
    What kinds of mitigating factors might reduce the impact of climate
    change?
  \item
    What kinds of adaptations might make it easier to live with climate
    change?
  \end{enumerate}
\item
  In the chapter on farming, pay attention to the discussion of
  productivity growth in the worlds' economies (basically this is the
  growth of \(g\), the per-capita GDP, in the Kaya identity, which we
  read about on p.~30 of Ch. 3, and which we'll study in greater detail
  when we look at decarbonizing the world's energy supply on
  Oct.~15--22.). How does growing productivity affect the way we look at
  climate change?
\item
  In the chapter on farming, why is figure 15 important to the
  discussion of adaptation and mitigation?
\item
  In the chapter on health impacts, don't try to get every detail but do
  try to get a sense of what the biggest climate related threats to
  health are likely to be and what kinds of adaptive things people could
  do to fight them as temperatures rise.
\item
  In the chapter on the oceans, there are two distinct threats:
  sea-level rise and ocean acidification. Get a feel for how each
  affects people's lives and what adaptations might be possible.
\end{itemize}

\hypertarget{wed.-feb.-26-how-will-climate-change-affect-our-lives-part-2}{%
\section{Wed., Feb.~26: How Will Climate Change Affect Our Lives? (Part
2)}\label{wed.-feb.-26-how-will-climate-change-affect-our-lives-part-2}}

\hypertarget{reading-21}{%
\subsection{Reading:}\label{reading-21}}

\hypertarget{required-reading-everyone-16}{%
\subsubsection{Required Reading
(everyone):}\label{required-reading-everyone-16}}

\begin{itemize}
\tightlist
\item
  Climate Casino, Ch. 10--12.
\end{itemize}

\hypertarget{reading-notes-16}{%
\subsubsection{Reading Notes:}\label{reading-notes-16}}

Mostly, skim chapters 10--11 and focus on reading chapter 12 carefully.

\begin{itemize}
\tightlist
\item
  What sectors of the economy are most vulnerable to climate change?
\item
  What parts of the world are most vulnerable?
\item
  Think about Nordhaus's distinction between managed and unmanaged
  systems: Does management affect vulnerability to climate change?
\item
  How has the world's economy changed in the last 60 years or so? Which
  sectors have become more important and which have become less
  important? What does that imply as we look ahead to the impacts of
  climate change 100 years from now?
\item
  How do economists estimate the damage climate change might cause to
  the world economy? How certain are they about these estimates? What
  are the biggest sources of uncertainty?
\item
  The section on ``A Risk Premium'' is especialy important. Understand
  what a risk premium is and how this figures into Nordhaus's thoughts
  about policy.
\end{itemize}

\hypertarget{fri.-feb.-28-policy-myths}{%
\section{Fri., Feb.~28: Policy Myths}\label{fri.-feb.-28-policy-myths}}

\hypertarget{reading-22}{%
\subsection{Reading:}\label{reading-22}}

\hypertarget{required-reading-everyone-17}{%
\subsubsection{Required Reading
(everyone):}\label{required-reading-everyone-17}}

\begin{itemize}
\tightlist
\item
  The Climate Fix, Ch. 2.
\item
  Climate Casino, Ch. 25.
\end{itemize}

\hypertarget{reading-notes-17}{%
\subsubsection{Reading Notes:}\label{reading-notes-17}}

In \emph{The Climate Fix}, Pielke addresses three what he calls three
myths of climate policy:

\begin{enumerate}
\def\labelenumi{\arabic{enumi}.}
\tightlist
\item
  We lack political will to do anything about climate change
\item
  We must trade off the economy for the environment
\item
  We have all the technology we need to solve the problem
\end{enumerate}

With respect to myth \#2, consider this quotation, from a story on
National Public Radio on October 2, 2010:

\begin{quote}
Republican pollster Frank Luntz says it's clear why the politics of
climate change are so different {[}in 2010{]} than they were in 2008.
``\emph{What has changed is that the American economy went to hell. And
when you ask voters are they more concerned about destroying their
environment over the next 100 years or rehabilitating their economy over
the next 100 weeks, they'll choose the economy over the environment any
day},'' Luntz says.
\end{quote}

As you read Chapter 2 of \emph{The Climate Fix}, critically assess
Luntz's arguments. Pay special attention to the ``\textbf{Iron Law},''
described on pp.~46--50. This will be a crucial piece of Pielke's
analysis throughout the book.

On pp.~51--58 of \emph{The Climate Fix}, Pielke writes about the idea of
``\textbf{stabilization wedges},'' introduced by Robert Socolow and
Stephen Pacala, in the context of dismissing myth \#3. The stabilization
wedge concept is very important in climate policy and we will discuss
them further later in the semester, so read these pages reasonably
carefully and keep in mind that Pielke's dismissal is just one opinion.
A number of scholars and policy experts agree with Pielke, but many
disagree as well. This is a topic on which it's important to think for
yourself.

After dismissing many myths, Pielke offers his own ideas for how to
approach climate policy.

In Chapter 25 of \emph{Climate Casino}, Nordhaus discusses what we know
about public opinion on climate change. Nordhaus shows that in the past
decade, we have seen an increasing divide between what scientists think
about the facts of climate change and what the public thinks, that this
widening gap has proceeded together with a growing ideological divide
between liberals and Democrats on one side and conservatives and
Republicans on the other. He observes on p.~311 that ``Climate change is
an area where the political leaders have led public opinion.''

Nordhaus then proposes a way for small-government political
conservatives, such as himself, to close the divide between the parties
and between scientists and the public.

As you read both Nordhaus's and Pielke's analyses and recommendations,
ask yourself what you find it persuasive, what you agree with, and what
you disagree with. This is a time to start thinking both about what
kinds of climate policies you would want to pursue and how you would
critically analyze them for their strengths and weaknesses.

\hypertarget{mon.-mar.-2fri.-mar.-6-spring-break}{%
\section{Mon., Mar.~2--Fri., Mar.~6: Spring
Break}\label{mon.-mar.-2fri.-mar.-6-spring-break}}

Spring Break, no class.

\hypertarget{mon.-mar.-9-the-kaya-identity-energy-use-efficiency-and-conservation}{%
\section{Mon., Mar.~9: The Kaya Identity: Energy Use, Efficiency, and
Conservation}\label{mon.-mar.-9-the-kaya-identity-energy-use-efficiency-and-conservation}}

\hypertarget{reading-23}{%
\subsection{Reading:}\label{reading-23}}

\hypertarget{required-reading-everyone-18}{%
\subsubsection{Required Reading
(everyone):}\label{required-reading-everyone-18}}

\begin{itemize}
\tightlist
\item
  The Climate Fix, Ch. 3.
\item
  Climate Casino, Ch. 14.
\end{itemize}

\hypertarget{reading-notes-18}{%
\subsubsection{Reading Notes:}\label{reading-notes-18}}

In both chapters, the focus is on what it would take to reduce \COO{}
emissions around the world.

Key concepts that you should understand are:

\begin{itemize}
\item
  We will be discussing national and global energy consumption in terms
  of \textbf{quads} (see p.~63 in Climate Fix). You should have a good
  feeling for how much a quad is and how many quads the US consumes.
\item
  The \textbf{Kaya Identity} and the factors that go into it (p.~71 of
  \emph{The Climate Fix}; you might also go back and quickly review
  pp.~19--23 of \emph{Climate Casino}, which we read for Aug.~28).:

  \begin{itemize}
  \tightlist
  \item
    Total \COO{} emissions (\(F\))
  \item
    Population (\(P\))
  \item
    Per-capita GDP (\(g\))
  \item
    Energy intensity of the economy (\(e\))
  \item
    Carbon intensity of the energy supply (\(f\))
  \end{itemize}
\item
  How would you make sense of the fact that the U.S. has a much greater
  \(F\) than India and a slightly smaller \(F\) than China, but a much
  smaller \(e\) and \(f\) than either India or China?
\item
  What trends have we seen over the past several decades in the energy
  intensity of the economy and the carbon intensity of the energy
  supply? (pp.~74--79 of \emph{The Climate Fix} and Fig. 3 on p.~22 of
  \emph{Climate Casino})
\item
  What is \textbf{primary energy consumption} and how does it differ
  from other kinds of energy consumption?
\item
  Why does Pielke argue that \textbf{energy dependence} leads to
  \textbf{energy insecurity}?
\item
  In \emph{Climate Casino}, look at Figure 23 and the table of carbon
  emissions on p.~159. How do different fuels compare in terms of carbon
  emissions?
\item
  In \emph{Climate Casino}, look at Table 6 and get a sense of what
  activities cause the most \COO{} emissions.
\item
  What does the Kaya Identity and the material from \emph{Climate
  Casino} suggest for where we should focus in our economy to reduce
  \COO{} emissions?
\end{itemize}

\hypertarget{wed.-mar.-11-reducing-carbon-emissions-bottom-up-approaches}{%
\section{Wed., Mar.~11: Reducing Carbon Emissions: Bottom-Up
Approaches}\label{wed.-mar.-11-reducing-carbon-emissions-bottom-up-approaches}}

\hypertarget{reading-24}{%
\subsection{Reading:}\label{reading-24}}

\hypertarget{required-reading-everyone-19}{%
\subsubsection{Required Reading
(everyone):}\label{required-reading-everyone-19}}

\begin{itemize}
\tightlist
\item
  The Climate Fix, Ch. 4.
\end{itemize}

\hypertarget{optional-extra-reading}{%
\subsubsection{Optional Extra Reading:}\label{optional-extra-reading}}

\begin{itemize}
\tightlist
\item
  Handout:
  \href{http://iopscience.iop.org/article/10.1088/1748-9326/4/2/024010/meta}{Roger
  A. Pielke, Jr., ``The British Climate Change Act: a critical
  evaluation and proposed alternative approach,'' Environmental Research
  Letters \textbf{4}, 024010 (2009).}.
\item
  Handout:
  \href{http://iopscience.iop.org/article/10.1088/1748-9326/4/4/044001/meta}{Rogar
  A. Pielke, Jr., ``Mamizu Climate Policy: An Evaluation of Japanese
  Carbon Emission Reduction Targets,'' Environmental Research Letters
  \textbf{4}, 044001 (2009)}.
\item
  Handout:
  \href{https://www-sciencedirect-com.proxy.library.vanderbilt.edu/science/article/pii/S1462901110001383}{Roger
  A. Pielke, Jr., ``An Evaluation of the Targets and Timetables of
  Proposed Australian Emissions Reduction Policies,'' Environmental
  Science \& Policy \textbf{14}, 20 (2011).}.
\end{itemize}

\hypertarget{reading-notes-19}{%
\subsubsection{Reading Notes:}\label{reading-notes-19}}

We will spend two days on this chapter. It is very involved and the
technical analysis both of this chapter and of Jacobson's proposal for
converting the US to 100\% wind, water, and solar power will be the
focus of a project in the lab. You will follow Pielke's methods to
analyze the prospects for the US, China, and a third country to convert
a large part of their economies to clean energy by 2050.

The optional extra reading comprises three articles by Roger Pielke, in
which he applies the methods he describes in this chapter to analyze
proposed decarbonization policies in the U.K., Japan, and Australia.

\hypertarget{fri.-mar.-13-reducing-carbon-emissions-top-down-approaches}{%
\section{Fri., Mar.~13: Reducing Carbon Emissions: Top-Down
Approaches}\label{fri.-mar.-13-reducing-carbon-emissions-top-down-approaches}}

\hypertarget{reading-25}{%
\subsection{Reading:}\label{reading-25}}

\hypertarget{required-reading-everyone-20}{%
\subsubsection{Required Reading
(everyone):}\label{required-reading-everyone-20}}

\begin{itemize}
\tightlist
\item
  The Climate Fix, Ch. 4.
\end{itemize}

\hypertarget{optional-extra-reading-1}{%
\subsubsection{Optional Extra Reading:}\label{optional-extra-reading-1}}

\begin{itemize}
\tightlist
\item
  Handout:
  \href{http://iopscience.iop.org/article/10.1088/1748-9326/4/2/024010/meta}{Roger
  A. Pielke, Jr., ``The British Climate Change Act: a critical
  evaluation and proposed alternative approach,'' Environmental Research
  Letters \textbf{4}, 024010 (2009).}.
\item
  Handout:
  \href{http://iopscience.iop.org/article/10.1088/1748-9326/4/4/044001/meta}{Rogar
  A. Pielke, Jr., ``Mamizu Climate Policy: An Evaluation of Japanese
  Carbon Emission Reduction Targets,'' Environmental Research Letters
  \textbf{4}, 044001 (2009)}.
\item
  Handout:
  \href{https://www-sciencedirect-com.proxy.library.vanderbilt.edu/science/article/pii/S1462901110001383}{Roger
  A. Pielke, Jr., ``An Evaluation of the Targets and Timetables of
  Proposed Australian Emissions Reduction Policies,'' Environmental
  Science \& Policy \textbf{14}, 20 (2011).}.
\end{itemize}

\hypertarget{reading-notes-20}{%
\subsubsection{Reading Notes:}\label{reading-notes-20}}

We will continue discussing this chapter and work examples in class of
Pielke's calculations.

The optional extra reading comprises three articles by Roger Pielke, in
which he applies the methods he describes in this chapter to analyze
proposed decarbonization policies in the U.K., Japan, and Australia.

\hypertarget{mon.-mar.-16-the-cost-of-reducing-emissions}{%
\section{Mon., Mar.~16: The Cost of Reducing
Emissions}\label{mon.-mar.-16-the-cost-of-reducing-emissions}}

\hypertarget{reading-26}{%
\subsection{Reading:}\label{reading-26}}

\hypertarget{required-reading-everyone-21}{%
\subsubsection{Required Reading
(everyone):}\label{required-reading-everyone-21}}

\begin{itemize}
\tightlist
\item
  Climate Casino, Ch. 14, pp.~157--165.
\item
  Climate Casino, Ch. 15.
\end{itemize}

\hypertarget{reading-notes-21}{%
\subsubsection{Reading Notes:}\label{reading-notes-21}}

Here, we focus on the cost of reducing emissions. You should skim the
material from Ch. 14 lightly and focus on reading Ch. 15 carefully.
Remember how, in the reading notes for Oct.~9, I warned you about the
potential confusion between ``mitigating factors,'' (the benefits of
global warming, such as reduced heating costs for people who live in
cold climates and longer growing seasons for farmers, which can offset
some of the harms) and ``mitigation,'' which means cutting greenhouse
gas emissions. Here, we are talking about the latter.

\begin{itemize}
\tightlist
\item
  Nordhaus distinguishes two kinds of economic analysis of the cost of
  mitigation: Top-down and bottom-up (pp.~174--76 and Fig. 25). What is
  the difference? Do you think one is more reliable than the other? Why
  or why not?
\item
  Nordhaus discusses two aspects of mitigating greenhouse gas emissions:
  the technical aspects that determine what a perfect policy that is
  efficiently implemented could do; and the human aspects, which cause
  policies to be imperfectly designed and inefficently implemented in
  the real world. Compare the two curves in Fig. 26.
\end{itemize}

\hypertarget{wed.-mar.-18-goals-of-climate-policy}{%
\section{Wed., Mar.~18: Goals of Climate
Policy}\label{wed.-mar.-18-goals-of-climate-policy}}

\hypertarget{reading-27}{%
\subsection{Reading:}\label{reading-27}}

\hypertarget{required-reading-everyone-22}{%
\subsubsection{Required Reading
(everyone):}\label{required-reading-everyone-22}}

\begin{itemize}
\tightlist
\item
  Climate Casino, Ch. 17.
\item
  The Climate Fix, Ch. 6.
\end{itemize}

\hypertarget{reading-notes-22}{%
\subsubsection{Reading Notes:}\label{reading-notes-22}}

\begin{itemize}
\tightlist
\item
  In these chapters, Pielke and Nordhaus offer different accounts of the
  history of international treaties and policies to manage climate
  change.
\item
  Much of the focus is on the United Nations Framework Convention on
  Climate Change (UNFCCC), signed in 1992 and ratified by all 193 member
  states of the United Nations.
\item
  The UNFCCC is legally binding on its signatories, and requires them to
  " stabiliz{[}e{]} greenhouse gases concentrations in the atmosphere at
  a level that would avoid dangerous anthropogenic interference with the
  climate system." A problem is that the Framework did not define what
  constituted ``dangerous anthropogenic interference,'' or spell out any
  specific actions that the signatories would have to take under the
  treaty.
\item
  In subsequent years, much of the world's scientific and climate policy
  elites arrived at a rough consensus that raising the average
  temperature of the earth by more than 2\degC{} relative to
  preindustrial temperatures would constitute dangerous interference.
  Both Nordhaus and Pielke present critical examinations of this
  judgment.
\item
  The details of implementing the pledge under UNFCCC (both defining
  'dangerous interference" and deciding on specific actions) was left to
  subsequent negotiations, and the signatory nations have met every year
  at ``conferences of parties'' (COPs) to hammer out details. The most
  important implementation agreement was a treaty signed in Kyoto in
  1998, but never ratified by the United States. Both Nordhaus and
  Pielke discuss the Kyoto treaty and its pros and cons.
\item
  As you read this history and the discussion of the goal of limiting
  warming to no more than 2\degC{} above pre-industrial temperatures,
  try to become familiar with the history and think critically about the
  kinds of policies that were pursued and those that were not given
  serious consideration.
\end{itemize}

\hypertarget{fri.-mar.-20-costs-and-benefits}{%
\section{Fri., Mar.~20: Costs and
Benefits}\label{fri.-mar.-20-costs-and-benefits}}

\hypertarget{reading-28}{%
\subsection{Reading:}\label{reading-28}}

\hypertarget{required-reading-everyone-23}{%
\subsubsection{Required Reading
(everyone):}\label{required-reading-everyone-23}}

\begin{itemize}
\tightlist
\item
  Climate Casino, Ch. 18.
\end{itemize}

\hypertarget{mon.-mar.-23-pricing-carbon}{%
\section{Mon., Mar.~23: Pricing
Carbon}\label{mon.-mar.-23-pricing-carbon}}

\hypertarget{reading-29}{%
\subsection{Reading:}\label{reading-29}}

\hypertarget{required-reading-everyone-24}{%
\subsubsection{Required Reading
(everyone):}\label{required-reading-everyone-24}}

\begin{itemize}
\tightlist
\item
  Climate Casino, Ch. 19.
\end{itemize}

\hypertarget{wed.-mar.-25-carbon-pricing-instruments}{%
\section{Wed., Mar.~25: Carbon Pricing
Instruments}\label{wed.-mar.-25-carbon-pricing-instruments}}

\hypertarget{reading-30}{%
\subsection{Reading:}\label{reading-30}}

\hypertarget{required-reading-everyone-25}{%
\subsubsection{Required Reading
(everyone):}\label{required-reading-everyone-25}}

\begin{itemize}
\tightlist
\item
  Handout:
  \href{/files/reading_handouts/Economics_of_Regulating_Greenhouse_Gases.pdf}{J.M.
  Gilligan, ``The Economics of Regulating Greenhouse Gases: Command \&
  Control, Emissions Trading, and Emissions Taxes,'' handout}.
\end{itemize}

\hypertarget{fri.-mar.-27-discounting-and-the-value-of-time}{%
\section{Fri., Mar.~27: Discounting and the Value of
Time}\label{fri.-mar.-27-discounting-and-the-value-of-time}}

\hypertarget{reading-31}{%
\subsection{Reading:}\label{reading-31}}

\hypertarget{required-reading-everyone-26}{%
\subsubsection{Required Reading
(everyone):}\label{required-reading-everyone-26}}

\begin{itemize}
\tightlist
\item
  Climate Casino, Ch. 16.
\end{itemize}

\hypertarget{required-for-grad-students-optional-for-undergrads}{%
\subsubsection{Required for Grad Students (optional for
undergrads):}\label{required-for-grad-students-optional-for-undergrads}}

\begin{itemize}
\tightlist
\item
  Handout:
  \href{http://www.econ.yale.edu/~nordhaus/homepage/homepage/nordhaus_stern_science.pdf}{W.
  Nordhaus, ``Critical Assumptions in the Stern Review on Climate
  Change,'' Science \textbf{317}, 201--202 (2007)}.
\item
  Handout:
  \href{http://science.sciencemag.org.proxy.library.vanderbilt.edu/content/317/5835/203.long}{N.
  Stern \& C. Taylor, ``Climate Change: Risk, Ethics, and the Stern
  Review,'' Science \textbf{317}, 203--204 (2007)}.
\end{itemize}

\hypertarget{optional-extra-reading-2}{%
\subsubsection{Optional Extra Reading:}\label{optional-extra-reading-2}}

\begin{itemize}
\tightlist
\item
  Handout:
  \href{https://link-springer-com.proxy.library.vanderbilt.edu/content/pdf/10.1007\%2Fs10584-008-9434-9.pdf}{J.
  Quiggin, ``Stern and His Critics on Discounting and Climate Change: An
  Editorial Essay,'' Climatic Change \textbf{89}, 195 (2008)}.
\end{itemize}

\hypertarget{reading-notes-23}{%
\subsubsection{Reading Notes:}\label{reading-notes-23}}

Time-discounting is one of the most contentious and controversial
aspects of the economics and ethics of climate change. Entire books have
been written about this, and one class will not do justice to the topic.

\begin{itemize}
\item
  As you read this chapter in Nordhaus, try to follow the distinctions
  Nordhaus draws between several aspects of discounting:

  \begin{itemize}
  \tightlist
  \item
    \textbf{Pure time preference:} If I offer you a choice of getting a
    free dinner at a nice restaurant sometime this month or getting the
    same dinner five years from now, you would almost certainly prefer
    to get the nice dinner this month. Given the choice of getting
    something nice now, or in the near future, versus having to wait a
    long time, most people don't want to wait.
  \item
    \textbf{Opportunity cost:} If I offered you the choice of \$100 now
    versus in five years, as opposed to the fancy meal, something else
    comes into the picture: If you got the money now, but chose not to
    spend it right away, you could invest it so that in five years, you
    would have more than \$100. Spending money today rather than
    investing it for the future produces an opportunity cost (missing
    out on the compounding interest), which contributes another piece to
    the problem of discounting and the value of time.
  \item
    \textbf{Fairness and economic growth:} Economic growth has
    dramatically reduced poverty around the world over the past
    centuries. The decline of poverty has been especially rapid in the
    past half-century. If this trend continues, people living a few
    centuries in the future will have average per-capita incomes much
    more than 10 times what the average person earns today. Thus,
    spending money today to reduce the costs of climate change for
    future generations might be like taking from the poor (today's
    generation) to benefit the rich (future generations).
  \end{itemize}
\item
  Graduate students should read the two articles in Science (and
  undergrads are welcome to read them if you're interested). Each is a
  bit less than two pages and they are very clear. These articles are a
  nice distillation of the kinds of economic/ethical arguments about
  what it means to be fair and just that are very common in
  environmental policy, and especially in climate policy.

  There are no easy answers, and the challenge has led philosopher
  Stephen Gardiner to call the problem of climate change ``a perfect
  moral storm.''
\item
  The optional article by the economist, John Quiggin, is a much more
  technical discussion of the ethical and economic conundrums that arise
  from reducing the value of life and suffering to a mathematical
  equation about time. It's very good, but it's purely optional and you
  should not feel obliged to read it if you are not really excited to
  nerd out about these things.
\end{itemize}

\hypertarget{mon.-mar.-30-the-limits-of-economic-approaches}{%
\section{Mon., Mar.~30: The Limits of Economic
Approaches}\label{mon.-mar.-30-the-limits-of-economic-approaches}}

\hypertarget{reading-32}{%
\subsection{Reading:}\label{reading-32}}

\hypertarget{required-reading-everyone-27}{%
\subsubsection{Required Reading
(everyone):}\label{required-reading-everyone-27}}

\begin{itemize}
\tightlist
\item
  Handout:
  \href{https://link-springer-com.proxy.library.vanderbilt.edu/article/10.1007/s10584-008-9433-x}{T.
  Barker, ``The Economics of Avoiding Dangerous Climate Change,''
  Climatic Change \textbf{89}, 173--194 (2008)}.
\item
  Handout:
  \href{https://link-springer-com.proxy.library.vanderbilt.edu/content/pdf/10.1007\%2Fs10584-008-9436-7.pdf}{C.
  Jaeger, H.-J. Schellnhuber, and V. Brovkin, ``Stern's Review and
  Adam's Fallacy,'' Climatic Change \textbf{89}, 207--218}.
\end{itemize}

\hypertarget{reading-notes-24}{%
\subsubsection{Reading Notes:}\label{reading-notes-24}}

The article by Barker, ``The Economics of Avoiding Dangerous Climate
Change,'' is long. Read the following portions:

\begin{itemize}
\tightlist
\item
  Section 1 (Introduction)
\item
  Section 2 (Traditional and New Economics): pp.~176--178 (through the
  paragraph in the middle of the page that ends ``\ldots{} but any
  implications for policy are heavily qualified and extensive lists are
  given of the dangers that are not or cannot be monetised,' and the
  single paragraph that starts at the bottom of p.~179 (''Contrast also
  the pre-Stern traditional conclusion \ldots{} assumptions about human
  welfare and behaviour needed to make the mathematics tractable.")
\item
  Section 3 (Uncertainty): pp.~181--183, through the first complete
  paragraph on p.~183 (``There are serious problems \ldots{} fundamental
  differences between them.'')
\item
  Section 4 (Economic Ethics): Skip the first few pages, which summarize
  arguments about discounting that we already covered in previous
  classes. Read pp.~185--187, starting at the bottom of p.~185 with
  ``Justice as a theory of ethics'' and reading through the end of the
  section.
\item
  Section 5 (Engineering and History). Skip this section.
\item
  Section 6 (Social Choice): read pp.~189--190. Read through the first
  complete paragraph on p.~190 (``In theory \ldots{} adaptation
  investments.'')
\item
  Section 7 (Toward a New Economics): Read this whole section.
\end{itemize}

\hypertarget{wed.-apr.-1-the-case-for-renewable-energy}{%
\section{Wed., Apr.~1: The Case for Renewable
Energy}\label{wed.-apr.-1-the-case-for-renewable-energy}}

\hypertarget{reading-33}{%
\subsection{Reading:}\label{reading-33}}

\hypertarget{required-reading-everyone-28}{%
\subsubsection{Required Reading
(everyone):}\label{required-reading-everyone-28}}

\begin{itemize}
\tightlist
\item
  Handout:
  \href{https://web.stanford.edu/group/efmh/jacobson/Articles/I/sad1109Jaco5p.indd.pdf}{M.Z.
  Jacobson and M.A.~Delucchi, ``A Path to Sustainable Energy by 2030,''
  Scientific American, Nov.~2009, pp.~58--65}.
\item
  Handout:
  \href{https://www.nytimes.com/2017/06/20/business/energy-environment/renewable-energy-national-academy-matt-jacobson.html}{E.
  Porter, ``Fisticuffs over the Route to a Clean-Energy Future,'' New
  York Times June 20, 2017, p.~B1}.
\item
  Handout:
  \href{https://www.washingtonpost.com/news/energy-environment/wp/2017/06/19/a-bitter-scientific-debate-just-erupted-over-the-future-of-the-u-s-electric-grid/}{C.
  Mooney, ``A Bitter Scientific Debate just Erupted over the Future of
  America's Power Grid,'' Washington Post June 19, 2017}.
\end{itemize}

\hypertarget{required-for-grad-students-optional-for-undergrads-1}{%
\subsubsection{Required for Grad Students (optional for
undergrads):}\label{required-for-grad-students-optional-for-undergrads-1}}

\begin{itemize}
\tightlist
\item
  Handout: \href{http://www.pnas.org/content/112/49/15060}{M.Z. Jacobson
  \emph{et al}., ``Low-cost solution to the grid reliability problem
  with 100\% penetration of intermittent wind, water, and solar for all
  purposes,'' PNAS \textbf{112}, 15060 (2015)}.
\item
  Handout: \href{http://www.pnas.org/content/114/26/6722}{C.T.M. Clack
  \emph{et al}., ``Evaluation of a Proposal for Reliable Low-Cost Grid
  Power with 100\% Wind, Water, and Solar,'' PNAS \textbf{114}, 6722
  (2017)}.
\item
  Handout: \href{http://www.pnas.org/content/114/26/E5021}{M.Z. Jacobson
  \emph{et al}., ``The United States Can Keep the Grid Stable at Low
  Cost with 100\% Clean, Renewable Energy in All Sectors, Despite
  Inaccurate Claims,'' PNAS **114*, E5021 (2017)}.
\end{itemize}

\hypertarget{reading-notes-25}{%
\subsubsection{Reading Notes:}\label{reading-notes-25}}

Pielke has argued that we don't have the technology necessary to abandon
fossil fuels. Jacobson and Delucchi disagree and present a proposal
here.

\begin{itemize}
\item
  ``A path to sustainable energy'' presents Jacobson and Delucchi's
  proposal to the general public. ``Low-cost solution to the grid
  reliabilty problem'' (optional for undergrads) provides up-to-date
  technical details for implementing this plan in the US.
\item
  ``Evaluation of a proposal'' (optional for undergraduates) is a fierce
  critique by a group of scientists and engineers. Jacobson \emph{et
  al.} reply to this critique in ``The United States can keep the grid
  stable.'' This exchange spawned a huge fight in the clean energy
  research community with accusations of dishonesty on all sides. The
  news stories ``Fisticuffs'' and ``A bitter scientific debate''
  describe this fight.
\item
  As you read Jacobson and Delucchi's proposal and its critics, think
  about both the technical and political aspects:

  \begin{itemize}
  \tightlist
  \item
    Are Jacobson and his co-authors persuasive when they argue that we
    have the technology that we need to quickly transition from fossil
    fuels to renewable energy?
  \item
    If we do have the technology, what political obstacles would you see
    to implementing this proposal?
  \end{itemize}
\end{itemize}

\hypertarget{fri.-apr.-3-the-case-for-nuclear-energy}{%
\section{Fri., Apr.~3: The Case for Nuclear
Energy}\label{fri.-apr.-3-the-case-for-nuclear-energy}}

\hypertarget{reading-34}{%
\subsection{Reading:}\label{reading-34}}

\hypertarget{required-reading-everyone-29}{%
\subsubsection{Required Reading
(everyone):}\label{required-reading-everyone-29}}

\begin{itemize}
\tightlist
\item
  Handout:
  \href{http://issues.org/32-2/a-roadmap-for-u-s-nuclear-energy-innovation/}{R.K.
  Lester, ``A Roadmap for U.S. Nuclear Energy Innovation,'' Issues in
  Science and Technology \textbf{32} (2) (Winter, 2016)}.
\item
  Handout: \href{http://issues.org/33-4/clean-energy-mind-games/}{D.
  Ropeik, ``Clean Energy Mind Games,'' Issues in Science and Technology
  \textbf{33} (4) (Summer, 2017)}.
\item
  Handout:
  \href{https://www.houstonchronicle.com/business/columnists/tomlinson/article/Nuclear-power-as-we-know-it-is-finished-11727465.php}{C.
  Tomlinson, ``Nuclear Power as We Know it Is Finished,'' Houston
  Chronicle Aug.~3, 2017}.
\end{itemize}

\hypertarget{reading-notes-26}{%
\subsubsection{Reading Notes:}\label{reading-notes-26}}

Nuclear power has the potential to provide abundant clean energy with no
greenhouse gas emissions. However, there are two separate problems that
nuclear power must contend with: First, it is very expensive. Several
years ago, the CEO of Excelon, the largest operator of nuclear power
plants in the U.S., said that nuclear power was more expensive than any
other source of energy except solar photovoltaic. Second, nuclear power
is politically controversial because of public fears about safety,
especially after the 2011 nuclear accident in Fukushima, Japan.

In these readings, ``Nuclear power as we know it is finished'' and ``A
roadmap for U.S. nuclear energy innovation'' discuss the economics and
the prospects for whether nuclear power can become economically
competitive. ``Clean energy mind games'' is written by an expert on risk
and discusses the safety issues and public fears.

\hypertarget{mon.-apr.-6-geoengineering-solar-radiation-management}{%
\section{Mon., Apr.~6: Geoengineering: Solar Radiation
Management}\label{mon.-apr.-6-geoengineering-solar-radiation-management}}

\hypertarget{reading-35}{%
\subsection{Reading:}\label{reading-35}}

\hypertarget{required-reading-everyone-30}{%
\subsubsection{Required Reading
(everyone):}\label{required-reading-everyone-30}}

\begin{itemize}
\tightlist
\item
  The Climate Fix, Ch. 5, pp.~117--132.
\item
  Climate Casino, Ch. 13. Read the whole chapter, but focus especially
  on pp.~152--156.
\item
  Handout:
  \href{http://science.sciencemag.org.proxy.library.vanderbilt.edu/content/325/5943/955}{G.C.
  Hegerl and S. Solomon, ``Risks of Climate Engineering,'' Science
  \textbf{325}, 955 (2009)}.
\end{itemize}

\hypertarget{reading-notes-27}{%
\subsubsection{Reading Notes:}\label{reading-notes-27}}

We will discuss geoengineering as an alternative to rapidly cutting
greenhouse gas emissions. For today, we will focus on \textbf{solar
radiation management}: Techniques for cancelling out the enhanced
greenhouse effect by blocking sunlight from reaching the earth.

\begin{itemize}
\item
  As you read Pielke, pay attention to his discussion of the big
  picture. What does he mean by a \textbf{technological fix}, and what
  does he think about technological fixes for environmental problems?
\item
  What are Daniel Sarewitz's criteria for successful technological
  fixes?
\item
  In the context of technological fixes, Pielke describes climate change
  as a ``wicked problem.'' This is a phrase with a specific meaning in
  public policy analysis. It comes from a 1973 paper,\footnote{H.W.J.
    Rittel and M.M. Weber,
    ``\href{https://doi.org/10.1007/BF01405730}{Dilemmas in a General
    Theory of Planning},'' Policy Sciences \textbf{4}, 155 (1973).}
  which defines ``wicked problems'' as possessing ten different
  properties, all of which make it very difficult, even impossible, to
  find satisfactory solutions. A few of these properties include: Wicked
  problems have high stakes, so it is unacceptable to choose a solution
  that proves ineffective. They are plagued by uncertainty, so no one
  can tell in advance whether a solution will work well. They invoilve
  important tradeoffs, so anything that makes a solution attractive to
  one constituency will make it unattractive to another. They are
  irreversible, so trial and error is not an effective approach for
  finding good solutions.

  As you read this chapter, think about how the problem of
  geoengineering the climate fits the criteria of a wicked problem.
\item
  What does Pielke think about solar radiation management in terms of
  Sarewitz's criteria?
\item
  Pielke begins this chapter by discussing geoengineering as a
  ``\textbf{Plan B}'' for climate policy. What does he mean by this?
\item
  Where Pielke talks about ``Plan B,'' Nordhaus describes geoengineering
  as ``\textbf{salvage therapy}'' for the planet. How does his account
  of solar radiation management compare to Pielke's? Where do the two
  agree and where do they disagree?
\item
  Nordhaus, as an economist, focuses a lot on costs, comparing the costs
  and the benefits of any policy. How does he asses the costs and
  benefits of geoengineering?
\item
  The short paper by Hegerl and Solomon presents solar radiation
  management geoengineering from the perspective of two top climate
  scientists. As you read their thoughts, which focus almost entirely on
  the scientific aspects, how do you think of this in comparison to the
  policy focus in Nordhaus's and Pielke's discussions?
\end{itemize}

\hypertarget{wed.-apr.-8-geoengineering-carbon-dioxide-management}{%
\section{Wed., Apr.~8: Geoengineering: Carbon Dioxide
Management}\label{wed.-apr.-8-geoengineering-carbon-dioxide-management}}

\hypertarget{reading-36}{%
\subsection{Reading:}\label{reading-36}}

\hypertarget{required-reading-everyone-31}{%
\subsubsection{Required Reading
(everyone):}\label{required-reading-everyone-31}}

\begin{itemize}
\tightlist
\item
  The Climate Fix, Ch. 5, pp.~132--142.
\item
  Climate Casino, Ch. 14, pp.~165--168.
\item
  Handout:
  \href{/files/reading_handouts/Dyson-GlobalWarming-2008.pdf}{F. Dyson,
  ``The Question of Global Warming,'' New York Review of Books, June
  2008}.
\end{itemize}

\hypertarget{reading-notes-28}{%
\subsubsection{Reading Notes:}\label{reading-notes-28}}

For today we will focus on a second class of geoengineering, called
``\textbf{air capture},'' (Pielke calls this ``\textbf{carbon
remediation}'') meaning reducing the atmospheric concentration of \COO{}
by filtering it out of the atmosphere and storing it permanently
somewhere.

\begin{itemize}
\tightlist
\item
  How do Pielke and Dyson feel about the prospect of air-capture
  geoengineering? (You know about Pielke; Dyson is a very famous
  physicist who is often discussed as a top prospect for a future Nobel
  Prize).
\item
  How does Pielke think air-capture fits Sarewitz's criteria for a
  technological fix?
\item
  At the end, when Pielke discusses the ``\textbf{moral hazard}''
  argument against geoengineering, what does he mean by ``moral
  hazard?'' What do you think about the moral hazard argument?
\item
  What are the implications of geoengineering technology for climate
  policy in general? How does the prospect of geoengineering shape
  Dyson's, Nordhaus's, and Pielke's support or opposition to more
  conventional climate policy that focuses on cutting greenhouse gas
  emissions?
\item
  What do \emph{you} think about geoengineering and how does the
  prospect of geoengineering shape your views of climate policy?
\end{itemize}

\hypertarget{fri.-apr.-10-pragmatism-and-climate-policy}{%
\section{Fri., Apr.~10: Pragmatism and Climate
Policy}\label{fri.-apr.-10-pragmatism-and-climate-policy}}

\hypertarget{reading-37}{%
\subsection{Reading:}\label{reading-37}}

\hypertarget{required-reading-everyone-32}{%
\subsubsection{Required Reading
(everyone):}\label{required-reading-everyone-32}}

\begin{itemize}
\tightlist
\item
  Climate Casino, Ch. 23.
\item
  The Climate Fix, Ch. 9.
\end{itemize}

\hypertarget{mon.-apr.-13-global-warming-gridlock}{%
\section{Mon., Apr.~13: Global Warming
Gridlock}\label{mon.-apr.-13-global-warming-gridlock}}

\hypertarget{reading-38}{%
\subsection{Reading:}\label{reading-38}}

\hypertarget{required-reading-everyone-33}{%
\subsubsection{Required Reading
(everyone):}\label{required-reading-everyone-33}}

\begin{itemize}
\tightlist
\item
  Handout:
  \href{/files/reading_handouts/Victor_GlobalWarmingGridlock_Chapter01.pdf}{David
  Victor, \emph{Global Warming Gridlock}, (Cambridge University Press,
  2011)}, Ch. 1.
\end{itemize}

\hypertarget{reading-notes-29}{%
\subsubsection{Reading Notes:}\label{reading-notes-29}}

I have posted Chapter 1 online. The entire book is
\href{https://catalog.library.vanderbilt.edu/discovery/fulldisplay?docid=cambookslvl10.1017/CBO9780511975714\&context=PC\&vid=01VAN_INST:vanui\&search_scope=MyInst_and_CI\&tab=Everything\&lang=en}{available
as an e-book} online through the Vanderbilt Library.

\hypertarget{wed.-apr.-15-beyond-gridlock-second-best-policies}{%
\section{Wed., Apr.~15: Beyond Gridlock: Second-Best
Policies}\label{wed.-apr.-15-beyond-gridlock-second-best-policies}}

\hypertarget{reading-39}{%
\subsection{Reading:}\label{reading-39}}

\hypertarget{required-reading-everyone-34}{%
\subsubsection{Required Reading
(everyone):}\label{required-reading-everyone-34}}

\begin{itemize}
\tightlist
\item
  Handout:
  \href{/files/reading_handouts/Beyond_Politics_Chapter_1.pdf}{M.P.
  Vandenbergh and J.M. Gilligan, \emph{Beyond Politics} (Cambridge,
  2017)}, Ch. 1.
\end{itemize}

\hypertarget{fri.-apr.-17-obstacles-and-perspectives}{%
\section{Fri., Apr.~17: Obstacles and
Perspectives}\label{fri.-apr.-17-obstacles-and-perspectives}}

\hypertarget{reading-40}{%
\subsection{Reading:}\label{reading-40}}

\hypertarget{required-reading-everyone-35}{%
\subsubsection{Required Reading
(everyone):}\label{required-reading-everyone-35}}

\begin{itemize}
\tightlist
\item
  Climate Casino, Ch. 26.
\item
  Handout:
  \href{/files/reading_handouts/Krugman-GamblingWithCivilization-2013.pdf}{P.
  Krugman, ``Gambling with Civilization,'' New York Review of Books,
  Nov.~7 2013}.
\end{itemize}

\hypertarget{mon.-apr.-20-review}{%
\section{Mon., Apr.~20: Review}\label{mon.-apr.-20-review}}

\hypertarget{reading-41}{%
\subsection{Reading:}\label{reading-41}}

No reading for today.

\hypertarget{notes-4}{%
\subsubsection{Notes:}\label{notes-4}}

No reading for today. We will review the semester and discuss climate
change and climate policy in the big picture.

\end{document}
